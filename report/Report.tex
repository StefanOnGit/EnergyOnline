\documentclass{scrartcl}
\usepackage[utf8]{inputenc}
\usepackage{amsmath,amssymb,amstext}
\usepackage{mathtools}

\usepackage[onehalfspacing]{setspace}

\begin{document}
	
	\title{Energy Online}
	\subtitle{}
	\author{J. Shah, M. Jordi, M. Stemmle, S. Oeschger, S. Zemljic}
	
	\maketitle
	\section{Introduction - Michael J. - arround 600 words}
    \paragraph{}
   The BIOTS program exists to push education for innovative new technologies into universities to help Switzerland o become one of the leading hubs in Blockchain and IoT space. (Source http://biots.org/about/)
   
    Since 2016, every year the BIOTS program offers three major verticals for the students to build their ideas and use-cases. In 2018, the focus was on sustainability issues, which are currently on the agenda of the UN or hot topics in industry. Challenges from different sponsors across the following verticals were solved:
    
    -- PICTURE biots2018\_verticals.PNG -- (Source http://biots.org/the-biotsphere/)
    
    This  report was created as a result of the EWZ. 
    
    \paragraph{}
    It has been decided, that all nuclear plants in Switzerland have to go off the grid until 2034. Currently 40\% of the electricity in Switzerland is provided by nuclear power plants. To be able to cover those losses, massive investments into renewable energy are needed. Large power plants alone will not be enough to cover the demands, thus small scale energy production, like private PV's will play a huge role.
    \paragraph{}
    Currently, energy surplus from private power plants (mainly PV) can only be sold to local energy provider for a fixed price. Combined with high entry barriers there is no real incentive to build private energy plants.
	To reach the goal of 0\% nuclear energy it is important to make private plants more lucrative and support them with subsidies. 
    
    Challenge 3 "Virtual Energy Storage" which consists out of the following problem:  "How might the blockchain technology help to develop a reliable and highly efficient virtual energy solution that gives anyone the possibility to produce, store and use his/her own energy anywhere and anytime? " (Source BIOTS\_ewz\_challenges.pdf )

	\section{Literature Review - Jig - arround 1100 words}
    
    \begin{itemize}
    \item Business Model https://www.eon-solar.de/eon-solarcloud
    \item Business Model https://www.senec-ies.com/tarife-services/senec-cloud/
    \item Certificats: http://www.bfe.admin.ch/themen/00612/00614/index.html?lang=en
    \item Blockchain and Energy https://hbr.org/2017/03/how-utilities-are-using-blockchain-to-modernize-the-grid
    \item Blockchain and Energy https://www.etla.fi/wp-content/uploads/ETLA-Working-Papers-43.pdf
    \item Blockchain and Energy http://ieeexplore.ieee.org/abstract/document/7589035/
    \item .........
    \end{itemize}
	
	\section{Conceptual Model - Simon arround 1100 words - Michael J. arround 500 words}
	
	\paragraph{}
	Our aim is to create a Decentralised Autonomous Organisation (DAO). The DAO issues energy tokens for energy produced, witch can be traded freely through the blockchain network. The tokens can then be used to receive electricity again. The DAO uses the surplus energy in the network to trade on the energy stock market. With the profit from trading, new renewable power plants are subsidized. The DAO's behaviour is completely controlled by smart contracts.
	
	\paragraph{}
	The final goal is nothing less, than a complete disruption of the energy market. Electricity could be freely traded peer to peer, without the need of a controlling instance. The limitations through cantonal borders, would fall away, even trade across national borders could be possible at a later point. Current energy providers would take the role of a hardware provider, ensuring the maintenance of the energy grid and maybe the smart meters financed through fees for the usage of the infrastructure. 
	
	\subsection{Initialization through ICO}
	
	\paragraph{}
	At the beginning, there will be an "Initial Coin Offering" (ICO), to get the initial funds for the DAO and to distribute the tokens. A major part of the tokens will be sold to investors and given to energy producers. A smaller part could be distributed for example to core developers or held back as reserves. This process would give the DAO its starting capital to start investing in renewable energy, and then trade on the energy stock market. 
	\subsection{Decentralised Autonomous Organisation DAO}
	
	
	\paragraph{}
	A DAO's behaviour is defined by smart contracts. This includes how to invest in further power plants  and trading behaviour. With the surplus energy from the producing members the DAO trades on the energy stock market, according to the smart contracts. With the profit made from trading, the DAO pays subsidies to build renewable power plants and thus enlarging it's network. Mining happens on a proof of stake implementation, to keep energy consumption in check. 

    
   \section{Solution Design - Stefan - arround 900 words}
   Coding: Architecture, modules etc.
   
   \section{Evaluation - Stefan - arround 200 words}
	Features, Bugs etc.
	
    \section{Conclusion/Outlook - Michael S. - arround 600 words}
	\paragraph{}
	A very important part are reliable measurements of energy production and consumption. It would be ideal, if smart meters could provide the data directly to the network. But it's highly possible that this does not work yet. As the trust in ewz is very high, and they already check energy consumption manually, this is a service ewz might provide.
	
	\paragraph{}
	Another problem might be on the legal front. The DAO is still a young concept with no laws existing yet. There is also the question, of how the certificates are handled in such a system.  
    
\end{document}